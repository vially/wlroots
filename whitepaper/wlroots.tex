\documentclass{article}
\usepackage{hyperref}
\usepackage[nohyphen]{underscore}
\newcommand{\code}[1]{\texttt{#1}}
\begin{document}
\title{Modular Wayland compositors with wlroots}
\author{Drew DeVault et al
    \thanks{Complete list available at \url{https://github.com/swaywm/wlroots/graphs/contributors}}}
\maketitle

\begin{abstract}
wlroots is a flexible and modular library upon which Wayland compositors can be
built. It is designed to maximize several factors: code reusability,
interoperability between Wayland compositors, and flexibility for use in novel
compositor designs.
\end{abstract}

\newpage
\tableofcontents

\newpage
\section{Background}

Wayland is a protocol used to communicate between user applications and a
\emph{compositor}. The compositor has ownership over input and output
resources -- typically displays connected over DVI, HDMI, etc, and keyboards,
mice, etc.

wlroots (stylized in lowercase, or as "wlr") is a subproject of the Sway
compositor. Originally, Sway was written on top of a similar library called wlc.
This library offered abstractions on top of DRM, GBM, GLES, libinput, and so on
-- allowing Sway to focus on building the software specific to the compositor.
However, with time it became clear that wlc's design is too limiting, and many
of Sway's long-term goals were difficult to achieve based on it.

\subsection{Alternatives}

Several alternative approaches to building compositors exist.\\

\textbf{libweston} is the most notable alternative. It is a subproject of the
Weston compositor, the reference Wayland compositor. However, it is generally
poorly suited to novel compositors that don't use the traditional desktop
paradigm. It is also less modular and difficult to break into several pieces.
Additionally, as a consequence of being refactored out of an existing
compositor, it inherits many design decisions of Weston itself, which may not be
amenable to other compositors.

\textbf{swc} and \textbf{wlc} are similar. They both strike a reasonable balance
between size and features, but both make decisions for the compositor that lock
them into particular designs and make novel compositors more difficult to
write. The advantage of their approach is that it is easy to use them to build a
simple working compositor - in a matter of minutes, not days or weeks.

Another approach is to simply write your compositor entirely in-house. This is
the approach favored by \textbf{mutter} and \textbf{kwin} of the GNOME and KDE
projects respectively. The advantage of this approach is the flexibility to
design your compositor in any way you please. However, lots of work is
duplicated between projects with this approach, which can lead to compositors
which are subtly incompatible with each other.

The approach taken by \textbf{wlroots} offers reusability and consistency
between compositors based on it, and offloads lots work from the compositor
implementation. This is accomplished without making design decisions on behalf
of the compositor implementation: we believe virtually any novel kind of
compositor can be built on top of wlroots.

The main disadvantage of our approach is that batteries are not included. More
work is required to get a working compositor compared to libweston or wlc.
However, you will not hit the limitations of these libraries, and will more
quickly arrive at a diverse featureset than with the kwin or mutter approaches.

\section{High-level Organization}

wlroots is organized into many discrete components that can be used separately
or replaced with your own implementations, and freely composed.

\subsection{Backends}

Backends are used to abstract access to input and output hardware. A backend,
once initialized, provides some number of \textbf{wlr_outputs} and
\textbf{wlr_input_devices}, the latter being an abstraction on top of types
like \textbf{wlr_pointer}, \textbf{wlr_keyboard}, \textbf{wlr_touch}, and so on.

Backends include \emph{drm} and \emph{libinput}, which abstract the
corresponding Linux subsystems; \emph{wayland}, which uses Wayland windows on
another running Wayland compositor; the similar \emph{x11} backend; \emph{rdp},
which uses the remote desktop protocol; and so on. An interface is provided so
that compositor implementations can supply their own backend implementations as
well.

Also available is the \emph{multi} backend, which acts as a container for many
backends. This is necessary, for example, to use DRM and libinput together. It
also allows you to add and remove backends at runtime. An example use-case:
adding an RDP backend at runtime.

\subsection{Types}

Types are at the heart of wlroots. They offer small, composable bits of
functionality, and include simple data structures like \textbf{wlr_box},
implementations of Wayland interfaces like \textbf{wlr_wl_shell}, and wlr tools
like \textbf{wlr_cursor}. Also included are types like \textbf{wlr_keyboard},
which are abstractions for backend types. Types like this are generally
accompanied by an \emph{interface}, which can be implemented by compositors and
integrated with other parts of wlroots.

\newpage
\section{Backends}

\subsection{Backend initialization}

There are two steps to setting up a backend -- first they are created, and then
they are started. During creation, the backend will obtain access to the
resources it needs, configure events on the event loop, test for required
features, and so on. During this stage, backends providing outputs are also
expected to configure their EGL context. Starting a backend will begin
presenting itself to the user and will start to create input and output devices
and pass them to the compositor. Backends are started with
\textbf{wlr_backend_start}, available in \code{wlr/backend.h}. Creation is more
subtle.

\subsection{Automatic selection}

The simplest way to obtain a backend is with \textbf{wlr_backend_autocreate},
available in \code{wlr/backend.h}. It will examine the runtime environment and
return the most suitable \textbf{wlr_backend}.

\subsection{Multi-backend}

A backend is available in \code{wlr/backend/multi.h} which can encapsulate many
other backends. Inputs and outputs added in these several backends can be
bubbled up through the multi-backend as if it were a single backend. Backends
can also be added to and removed from a multi-backend at runtime.

\subsection{DRM}

TODO

\subsection{Libinput}

TODO

\subsection{Wayland}

The Wayland backend, available in \code{wlr/backend/wayland.h}, allows you to
open Wayland windows on a different Wayland compositor and treat them as
\textbf{wlr_outputs} for nesting compositors. You may add \textbf{wlr_outputs}
dynamically at runtime with \textbf{wlr_wl_output_create} (and in fact you must
add at least one of these explicitly for the backend to be effective).

\subsection{X11}

TODO

\subsection{RDP}

TODO

\subsection{Headless}

TODO

\end{document}
